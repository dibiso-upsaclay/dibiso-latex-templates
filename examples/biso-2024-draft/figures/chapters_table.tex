\begin{longtable}{p{.4\linewidth}p{.35\linewidth}p{.15\linewidth}}
\caption{Liste des chapitres}
\label{tab_chapters}\\
\toprule
Titre du chapitre & Titre du livre & Éditeur \\
\midrule
Concurrence et régulation & Commentaire J. Mégret : L’énergie &  \\
La place du charbon dans l’Union européenne & Commerce transnational et industries extractives : entre singularité et pérennité d’un modèle &  \\
Démocratie environnementale : quelle(s) réalité(s) derrière les mots ? & Le droit économique de l’environnement : Acteurs et méthodes & Mare \& Martin \\
Ethique, science et gouvernance internationale en temps de pandémie & Sciences et pandémies : quelle éthique pour demain ? & éditions Erès \\
A la recherche de l’esprit de solidarité énergétique & Union européenne et solidarité(s) & Bruylant \\
Une certaine idée du service public culturel. Du 19ème au début du 20ème siècle & L’artiste, l’administrateur et le juge. L’invention du service public culturel. Le rôle du Conseil d’Etat, Actes du colloque des 26 et 27 novembre 2021, organisé par le Comité d’histoire du Conseil d’Etat et de la juridiction administrative et le Comité d’histoire du ministère de la culture & La rumeur libre éditions \\
Le Parlement comme contre-pouvoir sous la Ve République. Les enseignements des élections législatives de juin 2022 & Institutions et contre-pouvoir. Représentation, contrôle et ordre constitutionnel au prisme du droit constitutionnel comparé & Nihon Hyoron Sha \\
Le serment du chef de l’État & Le Serment. Perspectives juridiques contemporaines & Société de législations comparée \\
La 9e question mise au programme du Congrès international de droit comparé de 1900 & Mélanges en l’honneur du Professeur Ken Hasegawa & Mare \& Martin \\
Le permis de construire sur le domaine public : entre protection du domaine et simplification des autorisations d’urbanisme. À propos de l’article R. 431-13 du code de l’urbanisme & Droit de l’Aménagement de l’Urbanisme de l’Habitat, 27e éd. & Le Moniteur \\
Essais nucléaires et statut colonial de la Polynésie française : la rhétorique indépendantiste au sein de l’Organisation des Nations Unies & Le traitement juridique contemporain du fait nucléaire français en Polynésie française & Pedone \\
La réponse de l’Union européenne à la crise énergétique : unie dans le conflit ? & La conflictualité dans l’Union européenne : menace existentielle ou catalyseur d’intégration ? & Bruylant \\
A propos des accords bilatéraux de coopération récemment conclus avec trois Etats voisins (traités d’Aix-la-Chapelle, du Quirinal et de Barcelone) & Annuaire français de droit international (AFDI) &  \\
La typologie des régimes politiques à l’aune de la structure de l’exécutif & Les figures contemporaines du chef de l’État en régime parlementaire & Bruylant \\
La solidarité européenne à l’épreuve de la pandémie : bilan d’une stratégie vaccinale en clair-obscur & Ethique et gouvernance internationale de la recherche – Les enseignements de la pandémie de Covid-19 &  \\
L’utilisation du contrôle de proportionnalité par les juges européens en matière d’asile & Les juges européens face aux migrations & Anthémis \\
Préface & La déclaration Union européenne-Turquie, ambiguïtés et devenir d’un modèle de gestion des flux migratoires & Bruylant, collection Pratique(s) du droit international \\
\bottomrule
\end{longtable}
