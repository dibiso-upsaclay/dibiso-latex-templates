%% needed for the template to find the class dir when the class folder
%% is not stored in the same directory than the main tex file
\def\classdir{dibiso} % TODO: fix undefined classdir + improved explanations

\newcommand{\reportyear}{2024}
\newcommand{\labacronym}{XXX}
\newcommand{\labfullname}{Lorem ipsum dolor sit amet, consectetuer adipiscing elity}
\newcommand{\datafetchdate}{23/07/2025}

\newcommand{\collaborationsnb}{9007}
\newcommand{\institutionsnb}{2421}
\newcommand{\countriesnb}{86}
\newcommand{\oaworksperiod}{2020 - 2024}


\documentclass[french, 11pt]{dibiso/biso}

\title{Bilan \\ Science \\ Ouverte}

\author{Direction des bibliothèques, de l’information et de la science ouverte}

\date{Année \reportyear}

\subtitle{\textbf{Laboratoire \labacronym} \\
  \medskip
  \labfullname
}

% Remplacez Reporter Name par votre nom :
\reporter{Reporter Name}
% Remplacez reporter.email@example.com par votre email :
\reporteremail{reporter.email@example.com}


%%%%%%%%%%%%%%%%%%%%%%%%%%% COMMENT REMPLIR LE BISO %%%%%%%%%%%%%%%%%%%%%%%%%%%
%                                                                             %
% Commenter des parties à exclure :                                           %
% Pour commenter des sections que vous ne souhaitez pas inclure dans le       %
% rapport final, vous pouvez utiliser le symbole % au début de chaque ligne   %
% que vous souhaitez exclure. Cela peut vous servir à ne pas afficher des     %
% sections du rapports que vous ne jugez pas appropriées pour le laboratoire. %
%                                                                             %
%                                                                             %
% Créer une liste à puces :                                                   %
% Pour créer une liste à puces, utilisez l'environnement itemize. Voici un    %
% exemple :                                                                   %
% \begin{itemize}                                                             %
%     \item Premier élément de la liste                                       %
%     \item Deuxième élément de la liste                                      %
%     \item Troisième élément de la liste                                     %
% \end{itemize}                                                               %
%                                                                             %
%                                                                             % 
% Mettre du texte en gras :                                                   %
% Pour mettre du texte en gras, utilisez la commande \textbf{}. Par exemple : %
% Voici un exemple de \textbf{texte en gras}.                                 %
%                                                                             %
%%%%%%%%%%%%%%%%%%%%%%%%%%% COMMENT REMPLIR LE BISO %%%%%%%%%%%%%%%%%%%%%%%%%%%


\begin{document}

\renewcommand{\arraystretch}{1.5}


\maketitle

\tableofcontents

\pagebreak



\section{Introduction}

% Ecrire l'introduction ci-dessous :





% Fin de l'introduction

\pagebreak

\section{Liste des revues}

TOTO: add figure
%{
%  \footnotesize
%  \begin{longtable}{p{.55\linewidth}P{.08\linewidth}P{.11\linewidth}P{.11\linewidth}}
\caption{Liste des revues}
\label{tab_journals}\\
\toprule
Revue & Nombre de publications & Status des accès ouverts des publications & APC payés \\
\midrule
Volume 3A: Combustion, Fuels, and Emissions & \makecell{3} & \makecell{2 closed, \\ 1 green} & \makecell{} \\
Proceedings of the Combustion Institute & \makecell{7} & \makecell{6 other, \\ 1 hybrid} & \makecell{2 980 USD} \\
International Journal of Heat and Mass Transfer & \makecell{1} & \makecell{1 other} & \makecell{} \\
Chemical Engineering Research and Design & \makecell{2} & \makecell{2 green} & \makecell{} \\
Combustion and Flame & \makecell{4} & \makecell{4 other} & \makecell{} \\
Volume 3B: Combustion, Fuels, and Emissions & \makecell{1} & \makecell{1 green} & \makecell{} \\
Plasma Sources Science and Technology & \makecell{4} & \makecell{2 closed, \\ 2 green} & \makecell{} \\
Journal of Engineering for Gas Turbines and Power & \makecell{3} & \makecell{2 closed, \\ 1 green} & \makecell{} \\
Aerospace & \makecell{1} & \makecell{1 gold} & \makecell{1 600 CHF} \\
Applications in Energy and Combustion Science & \makecell{4} & \makecell{4 gold} & \makecell{2 000 USD, \\ 2 000 USD, \\ 2 000 USD, \\ 2 000 USD} \\
AIAA Journal & \makecell{1} & \makecell{1 green} & \makecell{} \\
Applied Physics B & \makecell{3} & \makecell{3 green} & \makecell{} \\
AIAA SCITECH 2023 Forum & \makecell{1} & \makecell{1 green} & \makecell{} \\
Computers \& Fluids & \makecell{1} & \makecell{1 green} & \makecell{} \\
International Journal of Multiphase Flow & \makecell{1} & \makecell{1 closed} & \makecell{} \\
CEAS Space Journal & \makecell{1} & \makecell{1 green} & \makecell{} \\
Frontiers in Physics & \makecell{1} & \makecell{1 gold} & \makecell{2 490 USD} \\
Biophysical Journal & \makecell{1} & \makecell{1 hybrid} & \makecell{2 500 USD} \\
Journal of Thermophysics and Heat Transfer & \makecell{1} & \makecell{1 green} & \makecell{} \\
Journal of Aerosol Science & \makecell{1} & \makecell{1 green} & \makecell{} \\
International Journal of Thermal Sciences & \makecell{1} & \makecell{1 green} & \makecell{} \\
Proceeding of International Heat Transfer Conference 17 & \makecell{1} & \makecell{1 green} & \makecell{} \\
\bottomrule
\end{longtable}

%}

% Ecrire un commentaire sur la liste des revues ci-dessous :





% Fin de la liste des revues

\pagebreak

\section{Liste des conférences}

\begin{figure}[!h]
  \includegraphics[width=\textwidth]{figures/conferences.pdf}
  \centering
  \caption{Conférences}
  \label{fig_conferences}
\end{figure}

% Ecrire un commentaire sur la liste des conférences ci-dessous :





% Fin de la liste des conférences


\pagebreak

\section{Liste des chapitres}

{
  \footnotesize
  \begin{longtable}{p{.4\linewidth}p{.35\linewidth}p{.15\linewidth}}
\caption{Liste des chapitres}
\label{tab_chapters}\\
\toprule
Titre du chapitre & Titre du livre & Éditeur \\
\midrule
Representations of Iceland in a religious context: of the Moravian Church as an outcome of cultural mobility ; Représentations de l'Islande dans un contexte religieux : l'Église morave comme résultat de la mobilité culturelle & Representations of the West Nordic Isles. Greenland – Iceland – Faroe Islands & Wachholtz Verlag \\
Usages et fonctions des structures à pierres chauffées du Toulousain au Néolithique. Nouvelles perspectives & Pêle-Mêle. Textes offerts à Jean Vaquer & Archives d'Ecologie Préhistorique \\
The Juillac Hoard (L’Isle-Jourdain, Gers): from discovery to study & Recent Discoveries of Tetrarchic Hoards from roman Britain and their wider Context & British Museum Press \\
Lejos de Francia, lejos de la violencia: la ambivalencia de una encuesta como expatriados en México & Violencia(s) y trabajo de campo en Mexico & CEMCA \\
POM -based Electrocatalysts for Carbon Dioxide Reduction & Applied Polyoxometalate‐based Electrocatalysis &  \\
Le verre d’époque carolingienne du site La Providence, rue Bernard, à Saintes (Charente-Maritime) & <em>MARE VITREUM</em>. Mélanges offerts à Danièle Foy & Éditions Mergoil \\
Ergonomia da atividade como oportunidade para a medicina participar do projeto /reprojeto do trabalho & Patologia do trabalho: o essencial, o novo e a prática. & Atheneu \\
La fouille du 13 bis rue des Ponts Chartrains : découverte d’une occupation agro-pastorale inédite du Ve/VIe au XIIe siècle dans le faubourg de Vienne à Blois (Loir-et-Cher) & L’archéologie des Ve-XIIe siècles en région Centre-Val de Loire, Actes des 41es Journées internationales de l’Association française d’Archéologie mérovingienne. Chartres (Eure-et-Loir) – 29 sept. au 2 oct. 2021, Vol. 2 & AFAM ; FERACF \\
Fuliginochronology: Reading Past Occupational Sequences in Carbonate Crusts & Mobile Landscapes and Their Enduring Places & Cambridge University Press \\
Indoor Laser-Based Wireless Communications & Free Space Optics Technologies in B5G and 6G Era - Recent Advances, New Perspectives and Applications [Working Title] & IntechOpen \\
\bottomrule
\end{longtable}

}

% Ecrire un commentaire sur la liste des chapitres ci-dessous :





% Fin de la liste des chapitres

\pagebreak

\section{Typologie de la production scientifique}

\begin{figure}[!h]
  \includegraphics[width=\textwidth]{figures/works_type.pdf}
  \centering
  \caption{Types de documents}
  \label{fig_doc_type}
\end{figure}

% Ecrire un commentaire sur la typologie de la production scientifique ci-dessous :





% Fin de la typologie de la production scientifique

\pagebreak

\section{Articles et Communication de congrès en accès libre} % Evolution sur une période de 5 ans (2020-2024)

\begin{figure}[!h]
  \includegraphics[width=\textwidth]{figures/open_access_works.pdf}
  \caption{Statut des accès ouverts des travaux sur la période \oaworksperiod}
  \label{fig_open_access_works}
\end{figure}

% Ecrire un commentaire sur les articles et communication de congrès en accès libre ci-dessous :





% Fin des articles et communication de congrès en accès libre

\pagebreak

\section{Carte des collaborations internationales}

{\collaborationsnb} collaborations parmi {\institutionsnb} institutions dans {\countriesnb} pays d'après la liste des articles avec un DOI dans HAL et les données de collaboration d'OpenAlex.

\begin{figure}[!h]
  \hspace{-.1\textwidth}\includegraphics[width=1.2\textwidth]{figures/collaboration_map_world.pdf}
  \caption{Collaborations internationales hors France - Données : publications renseignées dans HAL avec un DOI en utilisant les métadonnées d'OpenAlex}
  \label{fig_collab_map}
\end{figure}

% Ecrire un commentaire sur la carte des collaborations internationales ci-dessous :





% Fin de la carte des collaborations internationales

\pagebreak

\section{Carte des collaborations internationales - Focus sur l’Europe}

\begin{figure}[!h]
  \includegraphics[width=\textwidth]{figures/collaboration_map_europe.pdf}
  \caption{Collaborations internationales hors France en 2024 - Zoom sur l'europe - Données : publications renseignées dans HAL avec un DOI en utilisant les métadonnées d'OpenAlex}
  \label{fig_collab_map_europe}
\end{figure}

% Ecrire un commentaire sur la carte des collaborations internationales - Focus sur l’Europe ci-dessous :





% Fin de la carte des collaborations internationales - Focus sur l’Europe

\pagebreak

\section{Collaborations internationales par établissements}

\begin{figure}[!h]
  \includegraphics[width=\textwidth]{figures/collaboration_names.pdf}
  \caption{Collaborations internationales hors France en 2024 - Données : Nom des institutions renseignées dans HAL}
  \label{fig_collab_names}
\end{figure}

% Ecrire un commentaire sur les collaborations internationales par établissements ci-dessous :





% Fin des collaborations internationales par établissements

\pagebreak

\section{Collaborations (non académique) avec le secteur privé}

TODO: add figure

% Ecrire un commentaire sur les collaborations (non académique) avec le secteur privé  ci-dessous :





% Fin des collaborations (non académique) avec le secteur privé

\pagebreak

\section{Publications liées à des projets européens}

\begin{figure}[!h]
  \includegraphics[width=.8\textwidth]{figures/european_projects.pdf}
  \centering
  \caption{Projets Européens}
  \label{fig_eu_projects}
\end{figure}

% Ecrire un commentaire sur les publications liées à des projets européens ci-dessous :





% Fin des publications liées à des projets européens


\section{Publications liées à des projets ANR}

\begin{figure}[!h]
  \includegraphics[width=.8\textwidth]{figures/anr_projects.pdf}
  \centering
  \caption{Projets ANR}
  \label{fig_anr_projects}
\end{figure}

% Ecrire un commentaire sur les publications liées à des projets ANR ci-dessous :





% Fin des publications liées à des projets ANR

\pagebreak

\section{Données - jeux de données partagés}

% Ecrire un commentaire sur les données - jeux de données partagés ci-dessous :





% Fin des données - jeux de données partagés

\section{Logiciels partagés}

% Ecrire un commentaire sur les Logiciels partagés ci-dessous :





% Fin des Logiciels partagés

\section{Cahiers de laboratoire partagés}

% Ecrire un commentaire sur les cahiers de laboratoire partagés ci-dessous :





% Fin des cahiers de laboratoire partagés

\pagebreak

\section{Les atouts du Laboratoire dans son engagement pour la science ouverte}

% Ecrire un commentaire sur les atouts du Laboratoire dans son engagement pour la science ouverte ci-dessous :





% Fin des atouts du Laboratoire dans son engagement pour la science ouverte 


\section{Préconisations, pour aller plus loin}

% Ecrire un commentaire sur les préconisations, pour aller plus loin ci-dessous :





% Fin des préconisations, pour aller plus loin

\vfill

\paragraph{Rappel :} la loi pour la République Numérique permet aux auteurs travaillant dans une institution publique en France depuis 2016, de partager le postprint sur HAL, avec un embargo de 6 mois pour les disciplines STM et 1 an pour les SHS.

Des formations, des accompagnements peuvent être proposées par votre référent recherche pour guider les chercheurs. 




\makelastpagereport
 
\end{document}
